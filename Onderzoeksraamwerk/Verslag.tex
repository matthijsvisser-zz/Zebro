\documentclass[10pt,a4paper]{article}
\usepackage[utf8]{inputenc}
\usepackage{amsmath}
\usepackage{amsfonts}
\usepackage{amssymb}
\usepackage{graphicx}
\usepackage{color}
\usepackage{float}
\usepackage[dutch]{babel}
\usepackage{eurosym}
 
\definecolor{codegreen}{rgb}{0,0.6,0}
\definecolor{codegray}{rgb}{0.5,0.5,0.5}
\definecolor{codepurple}{rgb}{0.58,0,0.82}
\definecolor{backcolour}{rgb}{0.95,0.95,0.92}
\usepackage{listings}
\lstdefinestyle{cstyle}{
    backgroundcolor=\color{backcolour},   
    commentstyle=\color{codegreen},
    keywordstyle=\color{magenta},
    numberstyle=\tiny\color{codegray},
    stringstyle=\color{codepurple},
    basicstyle=\footnotesize,
    breakatwhitespace=false,         
    breaklines=true,                 
    captionpos=b,                    
    keepspaces=true,                 
    numbers=left,                    
    numbersep=5pt,                  
    showspaces=false,                
    showstringspaces=false,
    showtabs=false,                  
    tabsize=2
}
\graphicspath{ {./images/} }

\begin{document}
\begin{titlepage}
    \centering
    \vfill
    {\Large
    Swarming\\
   
    {\small Research framework}\\
        
        \vskip2cm
        {\small M. van Wilgenburg, W. Mukhtar, E. van Splunter, M. Siekerman, T. Zaal and M. Visser}\\
    }    
    \vfill
%    \includegraphics[width=1\textwidth]{WireS4}
    
    \vfill
    \vfill
\end{titlepage}

\newpage

\tableofcontents
\newpage


\section{Problem analysis}
Researcher have always been inspired by nature. When they looked at "social" insects like ants they discovered "swarming"\cite{swarmwiki} . The behaviour of one ant on its own seems illogical, but together they solve problems of great importance for the entire colony. These ants make us of the so called "trail laying" and "trail following" principle. Every ants lays a trail of pheromones, when a few ants walk back and forth to a food source. The one walking the shortest route will lay a more concentrated trail. The other ants will get attracted by the strongest trail, this way a positive feedback loop is created, which will make every ant walk the shortest route if given enough time. This is one example where relatively simple units, can achieve complex goals like path finding because they work together in a swarm. This principle is called "swarm intelligence"\cite{swarmintelligence}.

Swarming can have alot of up sides compared to the "classical" approach. A few are: quicker solution time, lower unit complexity and a greater fault tolerance\cite{swarmintelligence}. When for example one of the units breaks down, then will the other units still be able to complete the task. When this happens to one, more complex unit, this wont be the case. For these reasons its interesting to researching the applications of swarming.



\section{Context analysis} 
The Delft university of technology started a program to research the applications of swarming. In this program multiple universities work together to make this research possible. The idea of this program is that each project group contributes a small bit to reach the end result, which will be a working swarm of robots. The technology developed should be modular, so it can be easily used on other platforms.\\The programs focus lies on researching the applications of swarming on Mars. Its preferable that the technologies developed also work on Mars, but in some cases other technologies can be used to cut the cost. For example, for a simple proximity sensor on earth, an ultrasonic sensor would do just fine. But in the Mars atmosphere the ultrasonic waves get heavily dampened to the point where the sensor just wont work\cite{soundonmars}. The cheapest sensor that would work on Mars would be a lidar\cite{lidarmars}. While the average ultrasonic sensor costs around 2 \euro the cheapest lidar costs atleast 100 \euro. In this project the aim is to build one unit for around 200\euro, just one lidar sensor would be half of the robots budget. To proof the concept of swarming the robots don't need to be "Mars proof", so costs will be cut where possible.\\

\section{Problem definition}
Swarming intelligent systems are typically made up of simple agents(robots) interacting locally with one another and their environment. The group of individuals acting in such manner is referred to as a swarm\cite{swarmintelligence}. For a group of robots to qualify as a swarm-robotics the following criteria have to be met:

\begin{itemize}
    \item Autonomy – It is required that the individuals that make up   the swarm-robotic system are autonomous robots. They are able to        physically      interact with the environment and affect it.
    \item Large number – A large number of units is required
    as well, so the cooperative behavior (and
    swarm intelligence) may occur. The minimum number
    is hard to define and justify. The swarm-robotic
    system can be made of few homogeneous groups of
    robots consisted of large number of units. Highly heterogeneous 
    robot groups tend to fall outside swarm
    robotics.
    \item Limited capabilities – The robots in a swarm
    should be relatively incapable or inefficien on their
    own with respect to the task at hand.
    Scalability and robustness – A swarm-robotic
    system needs to be scalable and robust. Adding the
    new units will improve the performance of the overall
    system and on the other hand, loosing some units will
    not cause the catastrophic failure.
    \item Distributed coordination – The robots in a swarm
    should only have local and limited sensing and communication
    abilities. The coordination between the
    robots is distributed. The use of a global channel for
    the coordination would influence the autonomy of the
    units.
\end{itemize}

These criteria are a good indication of what makes a system swarm-robotic. But should not be used to determine whether a system is swarm-robotic or not. This is because some criteria are still somewhat phage\cite{swarmintelligence}.

Looking at these criteria we chose to define the problem into two sub-problems. One covers Autonomy, Large number and limited capabilities. And the other Distributed coordination. 

There are multiple aspects that should be taken into account. First the project focus will be building a swarming module. The features of this module will be discussed later. What's important is that the module should be "plug and play". What is meant by this is that the only in and output will be a communication protocol and a power input. This way the module can later be used on different platforms later in the program. To properly test the swarming module, a swarm of robots will be need to be developed.

\section{Doelstelling}
In samenwerking met de technische universiteit Delft en de Hogeschool van Amsterdam moet er een zwerm robots (zwerm) worden ontwikkeld die samen een taak uitvoeren. Zwermen is een breed begrip. Daarom wordt hieronder opgesomd, aan welke specificaties voldaan moet worden voordat er over zwermen gesproken mag worden. Tevens zal hieruit een duidelijk doel naar voren komen wat behaald moet worden wat betreft het laten samen werken van verschillende robots binnen een zwerm.

Een groep robots moet aan de volgende eisen voldoen om onder de categorie zwermen te vallen:

\begin{itemize}

\item Autonoom		-	Het is vereist dat de  individuele  robots  binnenin  een  zwerm   
systeem autonoom zijn. Ze moeten tevens over een fysieke interactie beschikken om de omgeving te waarnemen en eventueel aan te passen \cite{swarmintelligence}.

\item Een groot aantal	-	Een groot aantal is ook vereist om het samenwerkende gedrag
te bereiken. Het minimale aantal robots is lastig te definiëren. Een zwerm kan al gecre\"eerd worden met een paar simpele robots\cite{swarmintelligence}.
\item Gelimiteerd	-	De robots in een zwerm moeten individueel relatief incapabel
of ineffici\"ent zijn. Als een individuele robot de gehele taak van
de zwerm kan uitvoeren. Is in dit geval de inzet van de zwerm geheel nutteloos\cite{swarmintelligence}.

\item Schaalbaar		-	Een   zwerm    robots moet   schaalbaar  en   robuust   zijn. Het
toevoegen van nieuwe robots aan het gehele systeem moet de uitvoering van de taak over het geheel verbeteren. Tevens als er verschillende units binnen de zwerm verloren raakt mag dit niet tot catastrofale fouten lijden\cite{swarmintelligence}.

\item Co\"ordinaten	-	De robots   binnen een zwerm  moeten over een  gelimiteerde
communicatie en waarneming beschikken. De coördinaten tussen de verschillende robots gedistribueerd. Het gebruik van en globale communicatie kanaal voor de co\"ordinaten be\"invloed de autonomie van de verschillende units\cite{swarmintelligence}.
\end{itemize}

Het doel is om een zwerm robots te ontwikkelen die voldoen aan de bovenstaande criteria.

\section{Research-question}

The main question of this research is as following: \textit{"How can a minimum  of three robots operate in a swarm".} To give an answer to this question there are multiple subquestions to research first. 
 
%In dit onderzoek wordt de volgende hoofdvraag onderzocht:"Hoe kunnen er minimaal drie robots in een zwerm opereren?". Om de hoofdvraag te kunnen beantwoorden zijn antwoorden nodig op een aantal deelvragen.

\section{Sub-questions} 
The following questions need to be answered to come to a good conclusion to our research:

\begin{itemize}

    \item "What is swarming?"
    \begin{itemize}
        \item "What is the definition of swarming?"
        \item "How many robots are needed to create a swarm?"
        \item "How do the units communicate within the swarm?"
    \end{itemize}    
    \begin{itemize}
        \item "What software protocol should be used?"
        \item "What hardware protocol should be used?"
        \item "What is de minimal required communication speed?"
        \item "What hardware is needed to implement the communication?"
    \end{itemize}
    \item "What methods are there to propel the robot?"
    \begin{itemize}
        \item "What actuators can be used to drive the robot?"
        \item "How would the energy of the actuator be used to make the robot drive?"
        \item "What kind of effectors should be used?"
    \end{itemize}
    \item "What energy supply should be used to distribute the energy in the robot?"
    \begin{itemize}
        \item "How can the energy supply be managed?"
        \item "Is it possible to implement a recharge point?"
        \item "How can the energy supply be handled safely?"
    \end{itemize}
\end{itemize}
\newpage

\section{Specifications}
In this chapter the functional requirements for building a robot which can perform a task in cooperation with other robots will be defined. The specifications are divided into several features. The specifications to be achieved for swarming will be classified according to the MoSCoW-method. The corresponding specifications are further illustrated in this chapter. In the block diagram shown in Figure  \ref{fig:blockschematic}, the various modules are connected to each other.



\begin{figure}[h]
    \centering
    \includegraphics[width=1\textwidth]{blockschematic}
    \caption{simplified block diagram of the connection of the different subsystems within the robot }
    \label{fig:blockschematic}
\end{figure}



\subsection{Must haves}


M1 - Modular system architecture - Modules developed during this project must be modular so they can be used in future projects.\\\\
M2 - Units must be able to communicate with each other - The units in the swarm must be able to exchange information to properly  \\\\
M3 - Relative localization to other units \\
M4 - The communication needs to be wireless\\
M5 - Scalable swarm population\\


\subsection{Should haves}

S1 - Is able to carry a payload\\
S2 - The robot platform must be modular\\
M5 - The robot must be able to propel themselves omnidirectional\\

\subsection{Could haves}

C1 - Can move according to the principle of the ZebRo\\
C2 - Ability to self- charging / recharging of energy\\
C3 - Shows signs of intelligence\\
C4 - Is aware when the robot is upside down\\



\subsection{Won't haves}
\begin{itemize}
\item A master in the swarm network
\end{itemize}

\newpage

\section{sub-specifications}

\begin{table}[H]
\centering
\caption{Sensors}
\label{Sensors}
\begin{tabular}{|l|l|}
\hline
Module    & Sensors \\ \hline
Input  & Communication with the system          \\ \hline
Output & Communication with the system             \\ \hline
Functions   & \begin{tabular}[c]{@{}l@{}}measuring the distance between itself and the environment\\ To communicate the measured value to the controller\end{tabular}            \\ \hline
\end{tabular}
\end{table}

The sensors on the robot must create awareness of its surroundings so that he can anticipate. See Table \ref{Sensors}\\

\begin{table}[H]
\centering
\caption{Power supply}
\label{supply}
\begin{tabular}{|l|l|}
\hline
Module    & Power supply \\ \hline
Input  & Power supply            \\ \hline
Output & Communication with the system             \\ \hline
Functions   & \begin{tabular}[c]{@{}l@{}}measuring the voltage/ current of the Power supply\\ To send the measured value to the controller\end{tabular}            \\ \hline
\end{tabular}
\end{table}


It is important to measure the battery level to ensure that the robots will not be able to fall out. \\The measured value is send to the controller. See Table \ref{supply}\\

\begin{table}[H]
\centering
\caption{Wireless communication mudule}
\label{communication}
\begin{tabular}{|l|l|}
\hline
Module    & Wireless communication mudule \\ \hline
Input  & \begin{tabular}[c]{@{}l@{}}Communication with the system\\ Communication with the other robots in the swarm          
\end{tabular}   \\ \hline
Output & \begin{tabular}[c]{@{}l@{}}Communication with the system\\ Communication with the other robots in the swarm         
\end{tabular}   \\ \hline
Functions   & \begin{tabular}[c]{@{}l@{}}establishes the communication between the modules swarm \\ Obtains information regarding the distance between the different robots\end{tabular}            \\ \hline
\end{tabular}
\end{table}


To communicate between different units a communication network should be set up with a common protocol . In addition, the protocol must be able to provide additional information such as localization. See Table \ref{communication} \\

\begin{table}[H]
\centering
\caption{effectors}
\label{Effectors}
\begin{tabular}{|l|l|}
\hline
Module    & Effectors \\ \hline
Input  & motor drivers  \\ \hline
Output & mechanical power   \\ \hline
Functions  & Drives the robot , and is able to turn on command            \\ \hline
\end{tabular}
\end{table}

The operator must , in addition to the functioning in the testenvironment , also function well on Mars . See Table \ref{Effectors} \\

\begin{table}[H]
\centering
\caption{Central control unit}
\label{control}
\begin{tabular}{|l|l|}
\hline
Module    & Central control unit \\ \hline
Input  & All information from the system  \\ \hline
Output & All control of the modules in the robot \\ \hline
Functions  & All the communication comes together and controls the inputs and outputs            \\ \hline
\end{tabular}
\end{table}

The central control unit handles all communication between the various modules that are present on the robot. See Table \ref{control}

\newpage

\section{Bibliografie}
\bibliography{references}
\bibliographystyle{IEEEtran}



\end{document}