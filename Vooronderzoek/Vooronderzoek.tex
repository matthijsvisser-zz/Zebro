\documentclass[10pt,a4paper]{article}
\usepackage[utf8]{inputenc}
\usepackage{amsmath}
\usepackage{amsfonts}
\usepackage{amssymb}
\usepackage{graphicx}
\usepackage{color}
\usepackage{float}

\usepackage{eurosym}
 
\definecolor{codegreen}{rgb}{0,0.6,0}
\definecolor{codegray}{rgb}{0.5,0.5,0.5}
\definecolor{codepurple}{rgb}{0.58,0,0.82}
\definecolor{backcolour}{rgb}{0.95,0.95,0.92}
\usepackage{listings}
\lstdefinestyle{cstyle}{
    backgroundcolor=\color{backcolour},   
    commentstyle=\color{codegreen},
    keywordstyle=\color{magenta},
    numberstyle=\tiny\color{codegray},
    stringstyle=\color{codepurple},
    basicstyle=\footnotesize,
    breakatwhitespace=false,         
    breaklines=true,                 
    captionpos=b,                    
    keepspaces=true,                 
    numbers=left,                    
    numbersep=5pt,                  
    showspaces=false,                
    showstringspaces=false,
    showtabs=false,                  
    tabsize=2
}
\graphicspath{ {./images/} }

\begin{document}
\begin{titlepage}
    \centering
    \vfill
    {\Large
    Zwermen\\
   
    {\small Onderzoeksraamwerk behorende bij het innovatielab project}\\
        
        \vskip2cm
        {\small M. van Wilgenburg, W. Mukhtar, E. van Splunter, M. Siekerman, T. Zaal en M. Visser}\\
    }    
    \vfill
%    \includegraphics[width=1\textwidth]{WireS4}
    
    \vfill
    \vfill
\end{titlepage}

\newpage

\tableofcontents
\newpage


\section{Protocol}
\subsection{Hardware}

For the internal communication between the different systems there are multiple protocols to chose from to transmit data. every subsystem with a microcontroller should be able to communicate with the rest of the system. Because every system should be able to share their information it is necessary to have a multimaster system which allows every system to share their information without being dependent on a single master. To see which protocols are viable multiple protocols will be compared.

Building a modular robot consisting of subsystems, there must be a method to let these "\textit{subsystems}" communicate with each other to exchange relevant information. 

Every subsystem features a micro-controller which must be capable of transmitting and receiving data.  

\subsubsection{TWI}
TWI staat voor two wire interface en lijkt sterk op het protocol $I^{2}C$ van Phillips. zoals de naam al doet vermoeden bestaat de bus uit 2 aansluitingen, één voor de data en één voor de clock \ref{I2C bus}
\subsubsection{SPI}
\subsubsection{UART}
\subsection{Software}


\newpage
\section{Localization}
Building a swarm of robots, location awareness may be required to ensure an more efficient operation. Implementing location awareness into a swarm of robots could result in better cooperation. For example: exploration based tasks can be executed on a much lower time scale. With location awareness the swarm can spread out evenly making sure that every part of the perimeter will be explored thoroughly. 

With location awareness it is also possible to divide the swarm into multiple groups. Because the location of every individual is available these individual groups can be formed very quickly by selecting the nearest unit. This also enables quick assists when an individual robot gets stuck or needs to execute a task that require more than one unit.

To determine position, not only distance is required, to determine relative position the angle is also required to determine from which angle the units are relative to yourself. However this can result in multiple solutions, the angle could be mirrored.

\begin{figure}[H]
\centering
\includegraphics[width=0.9\textwidth]{drolletje.pdf}
\caption{Stationary robot (yellow) cannot compute the relative position of the moving robot (green), since all distance measurements (dashed lines) are invariant to rotations around the stationary robot.\cite{Angle}}
\label{Angle}
\end{figure}

Therefore, in order to leverage the previous trilateration procedure requires coordinating the motion of the robots in a manner that gives every robot a chance to move and ensures that when a robot is moving its neighbors remain stationary. This can also be achieved by the use of stationary anchors.\cite{Angle}

A more efficient solution would be to retrieve the relative orientation opposite to each other. In combination with the relative distance every robot knows the exact relative position of all the other robots in range of the network.
\newpage

\subsection{Relative Distance}
Relative distance can be measured using various methods. The most common methods are \textit{"Received Signal Strength Indication"} and \textit{"Time Of Flight"} abbreviated as RSSI and TOF. 

\subsubsection{Received Signal Strength}
A method to determine distance can be implemented using Received Signal Strength abbreviated as "RSSI". Typically RSSI is a measure of dBm, which is ten times the logarithm of the ratio of the power (P)
at the receiving end and the reference power (Pref). Power at the receiving end is inversely proportional to the
square of distance.\cite{RSSI} Hence RSSI could potentially be used as an indicator of the distance
at which the sending unit (robot) is located from the receiving unit.\cite{RSSI}

RSSI is defined as ten times the logarithm of the ratio of power of the received signal
and a reference power (for example 1mW). It is known that power dissipates from its source
as it moves further out. As mentioned earlier the relationship between power and distance is that power is inversely
proportional to the square of the distance travelled. \cite{RSSI}

In theory it seems as a suitable implementation for the purpose, determining relative distance between
different units. However various test results have shown that RSSI is only reliable under certain circumstances.
Test results have shown that if the orientation of the sending or receiving unit changes it becomes very unreliable.
The graphs below show the reliability of RSSI when the direction is being altered.\cite{RSSI}

\begin{figure}[H]
\centering
\includegraphics[width=0.9\textwidth]{North.pdf}
\caption{Distance versus RSSI plot showing inverse non linear
relationship.\cite{RSSI}}
\label{North}
\end{figure}

\begin{figure}[H]
\centering
\includegraphics[width=0.9\textwidth]{West.pdf}
\caption{Distance versus RSSI plot showing lack of
reliability when tested in a different direction.\cite{RSSI}} 
\label{West}
\end{figure}

\begin{figure}[H]
\centering
\includegraphics[width=0.9\textwidth]{East.pdf}
\caption{Distance versus RSSI plot showing lack of
reliability when tested in a different direction.\cite{RSSI}} 
\label{East}
\end{figure}
\newpage


Even though RSSI was very promising in many studies, through multi directional experiments it became clear that even under ideal conditions with weather and interference controlled, the RSSI data could not be relied upon. At times the data was correct showing the expected inverse square relation with distance and at other times it didn't.\cite{RSSI}


\subsubsection{Time Of Flight}
Another common method to determine relative distance is Time Of Flight abbreviated as "TOF". Time Of Flight describes a method that measures the time that an particular wave takes to travel a distance trough a medium. The wave can be acoustic, electromagnetic and light.

For relative distance measurements Time Of Flight is often combined with Time Difference Over Arrival abbreviated as "TDOA". Combining these methods enables relative distance measurements to specific units/nodes. TDOA measures the time difference between the transmission and arrival of a wave. With the velocity of propagation times the travelled time de relative distance between two or more nodes can be determined. However TOF also features certain limitations being clock synchronization, noise, sampling and multipath channel effects.\cite{TOF} 

\textit{Clock Synchronization}
TOF ranging systems need to estimate the time of transmission and arrival and comparing this with a common time reference.\cite{TOF} If the different time references are not perfectly synchronized a time offset error occurs. With the faster the propagation speed, the greater the error in distance. 

\textit{Sampling}
Using Light or electromagnetic waves to determine the relative distance can add several limitation. Due to the high velocity propagation (being nearly equal to equal of the velocity propagation of light) sampling this signal requires very high clock rates to do so. It would take a 15GHz clock to achieve one centimetre resolution.\cite{Arduino}

in the 1960's, Tektronix invented a sampling method with a oscilloscope that could measure signals in the GHz range without using high speed clocks. This method is called Sequential Equivalent Time Sampling abbreviated as "SETS". The SETS acquires one sample per trigger see figure \ref{SETS}. When a trigger is detected a sample is taken after a very short, but well defined delay. When the next trigger occurs a small time increment "Delta T" is added to this delay and the digitizer takes another sample. This process is repeated many times with "Delta T" added to each previous acquisition, until the time windows is filled.\cite{SETS} In figure \ref{SETS2} another example of an sinusoidal signal is displayed using the SETS method. 

Using the SETS method enables the use of waveform with an velocity propagation nearly equal to equal to the speed of light. However implementing a advanced SETS circuit is required to reach accuracy of less than 1 meter indoors.\cite{TOF} However using waveforms with a high velocity propagation enables faster data transfers and a larger range.\cite{TOF}

\begin{figure}[H]
\centering
\includegraphics[width=0.9\textwidth]{SETS.png}
\caption{In sequential equivalent time sampling a singel sample is taken for each recognized trigger after a delay which is incremented after each cycle.\cite{SETS}} 
\label{SETS}
\end{figure}

\begin{figure}[H]
\centering
\includegraphics[width=0.9\textwidth]{SETS2.png}
\caption{Example equivalent time sampled signal.\cite{SETS}} 
\label{SETS2}
\end{figure}
\newpage

\subsection{Conclusion}
Different studies have shown that Received Signal Strength can theoretically measure distance accurately. However RSSI seems to be very unreliable in many circumstances which leads to not being viable for distance measurements with moving robots. 

Time Of Flight has been very promising, however using electromagnetic signals (For example RF, WI-FI or Bluetooth) the acceptable accuracy starts at from a minimum range of 1-2 meters which leads to unreliable to no useful information at all in the range of 1-2 meters. This minimum range could be narrowed using an Sequential Equivalent Time Sampling Circuit, however such a circuit is complex to develop. However using waveforms with a lower velocity propagation can lead to accurate close range measurements. In this case Time Of Flight suits better for the required use, being more accurate and reliable see figure \ref{RSSIvsTOF}.\cite{TOF}

\begin{figure}[H]
\centering
\includegraphics[width=0.9\textwidth]{RSSIvsTOF.pdf}
\caption{Accuracy RSSI vs TOF.\cite{TOF}} 
\label{RSSIvsTOF}
\end{figure}

Determining relative distance does not lead to localization to do so a angle is required. 
\newpage

\section{Onderzoek}
Hier tekst plaatsen

\subsection{Propulsion,Actuators and Effectors}

In this chapter the sub-questions about the propulsion, actuators and effectors are researched and answered. First of all, the possible types of propulsion will be researched. Then the various kinds of actuators with a suitable effector will be considered. Depending on the situation different types of propulsion is used. For example a robot that operates in an environment with water would not be implemented with DC motor drivers and tyres. The robot build in this research should be able to operate on Mars in the future. Therefore the robot cant get stuck easily since that will cost millions, since the robot cannot complete his mission. Since swarming and not propulsion is the main goal of this research a simple way of locomotion would suffice. However there should be the possibility to implement a different kind of propulsion in the future, so that the robot can be used on Mars. 


%In dit hoofdstuk worden de deelvragen over de aandrijving, actuatoren en effectoren beantwoord. Allereerst zal er onderzocht worden welke soorten aandrijving er mogelijk zijn.Vervolgens zal er worden gekeken welke actuatoren er gebruikt kunnen worden en hoe de energie van de actuator gebruikt kan worden om de robot voort te bewegen. Als laatste wordt er gekeken welke effectoren nodig zijn. Robots worden op veel verschillende manier aangedreven, afhankelijk van de situatie.Bijvoorbeeld een robot die op het water moet opereren zal niet vaak met DC motoren met wielen worden ge\"implementeerd. In dit onderzoek kijken we naar een robot die in de toekomst mogelijk naar mars gestuurd kan worden. Dan kan de robot niet zomaar vast komen te zitten, aangezien er dan miljoenen voor niks verloren gaan. Omdat er eerst met een vereenvoudigde werkelijkheid wordt gewerkt is het vooral belangrijk dat de robot eenvoudig kan voortbewegen. Wel moet er ruimte zijn om later een andere module te kunnen aansluiten zodat hij ook op mars zich zou kunnen voortbewegen. 

\subsubsection{Effectors}
The way a robot moves is determined by its effector with his corresponding actuator. Possible types of effectors are:

%Door een effector aan een actuator de koppelen kan een robot zich voortbewegen. Mogelijke effectoren kunnen zijn:

\begin{itemize}
\item Wheels
\item Leg(s)
\item Catepillar tracks
\item Air cushion  
\item Propeller 
\item (Superconductor)
\end{itemize}

Each of these effectors have pro's and con's in different situations. Tyres are easily implemented, but there are various ways to implement these. The amount of tyres used will affect the handling of the robot. With a minimum two tyres, movement can already be realized. The downside is that driving with two wheels is not stable, only with algorithms that can balance the robot stable movement can be accomplished. It would be easier to implement more than two wheels so that the robot is more stable and movement is easier. 


  
%hier gebleven met vertalen
Elke van deze effectoren hebben voor en nadelen in verschillende situaties.\\Wielen zijn eenvoudig te implementeren, er kunnen verschillende aantallen wielen worden aangebracht. Het aantal wielen heeft wel invloed op het rijgedrag van de robot. Met twee wielen kan er worden gereden, mits de robot in balans wordt gehouden. Dit kan worden gedaan door middel van een zwenkwiel, of door een regelsysteem te bedenken dat de twee wielen de robot in balans houden. Dit laatste kost veel energie en is niet eenvoudig. Met drie wielen staat de robot al een stuk stabieler. Het nadeel hiervan is dat sturen soms moeizaam kan zijn. Met vier wielen is de robot nog stabieler en kan er op verschillende manieren worden gestuurd. De wielen kunnen tegengesteld worden aangedreven waardoor de robot draait. Dit kost wel meer energie dan nodig is aangezien er extra wrijving ontstaat. Een optie is de voorste wielen te laten draaien, waardoor de robot in de richting van de wielen zal voortbewegen. Zoals bij een auto het geval is. Dit is weer lastiger te implementeren aangezien er een soort van stuurmechanisme moet worden bedacht. 

\subsubsection{Actuatoren}

De keuze van de actuator hangt sterk samen met de effectoren. <3

\subsubsection{Aandrijving}

\section{Swarm Communication}
Each member of the swarm population needs to communicate with the rest of the swarm. To give the members full freedom of movement the communication also needs to be wireless. It is important that the population size of the swarm is dynamic, because of the scalability of the swarm. As mentioned earlier a swarm does not have a master. So each member needs to



There exist several techniques to accomplish this method 
Examples

To create an network with those characteristics a mesh network (MN) is needed. An mesh network relies on communication nodes in which data is distributed. All of the communication nodes cooperate in the distribution of data in the network. Each node of the network is connected to one or more nearby in range node(s) and forwards data on behalf of the other nodes. 

A wireless mesh network (WMN) is a dynamically self-organized and configured network. Each node in the network is able to create an ad hoc network to maintain the mesh connectivity. These networks are able to reroute around nodes that have been lost. 

%\begin{figure}[h]
%    \centering
%    \includegraphics[width=1\textwidth]{WMN}
%    \caption{An example of a wireless mesh network with 5 clients and a internet connection. The %connections made in the example are random.}
%    \label{fig:WMN}
%\end{figure}


$https://en.wikipedia.org/wiki/Mesh_networking$
$https://en.wikipedia.org/wiki/Wireless_ad_hoc_network$



For close range obstacle avoidance, the robots will need some kind of proximity sensing technique. The sensor needs to differentiate between different "treats" for the robot to function properly. For example; when the sensor detects a small hill, which the robot can move over it should not trigger the robot to move away from it. But when it detects a big obstacle it should trigger the robot to move away. This is important because the robots will eventually be expected to move across rough terrain (like Mars). In the next paragraphs some different proximity sensor techniques will be discussed. Price is also a important factor while comparing these techniques. Multiple robots must be made on a tight budget, so costs should be cut where possible.\\


\subsubsection{Ultrasoon sensor}
Een veel gebruikte proximity sensor op simpele robots is de ultrasoonsensor. Dit komt omdat deze in vergelijking tot andere sensoren een goedkope oplossing is. Een ultrasoon sensor stuurt een ultrasoon geluid uit dit geluid wordt teruggekaatst tegen het object en vervolgens weer opgevangen door de sensor. De tijd die het geluid er over gedaan heeft wordt gemeten. Met een simpel rekensommetje wordt vervolgens de afstand bepaald. Een probleem met de ultrasoon sensor is dat het moeilijk is om te discrimineren tussen verschillende objecten. Hierdoor wordt het moeilijk om onderscheid te maken tussen bijvoorbeeld een heuveltje (waar de robot wel overheen kan) en een obstakel die de robot moet ontwijken. Door de atmosfeer op mars wordt geluid ernstig gedempt hierdoor is ultrasoon niet implementeerbaar op mars \cite{soundonmars}. Voor testen op aarde is het echter wel een goed werkend en bewezen techniek.

\subsubsection{Lidar sensor}
Lidar staat voor “Light detection and ranging”. Deze sensor werkt volgens het zelfde principe als de ultrasoon sensor. Het is al bewezen dat de lidar techniek werkt op Mars. Met de Phoenix missie is er een lidar systeem gebruikt om wolken en atmosferische stof te meten\cite{lidarmars}. Het licht van de laser kon wolken meten op meerdere kilometers hoogte. Omdat licht zich zeer snel verplaatst, moet de elektronica in een lidar sensor ook snel zijn. Hierdoor is de prijs in vergelijking tot de ultrasoon sensor hoog.

\begin{figure}[!ht]

  \centering
      \includegraphics[width=0.5\textwidth]{voelsprieten.jpg}
  \caption{Robot met mechanische proximity sensor}  \label{voelspriet}
 
\end{figure}

\subsubsection{Mechanische sensor}
Eén mechanische proximity sensor bestaat meestal uit twee onderdelen. Dit zijn de actuator, dit is meestal een simpele druk schakelaar. En de arm of "bumper", dit is het deel dat het object aanstoot en zorgt dat de mechanishe kracht op de schakelaar wordt uitgeoefend figuur \ref{voelspriet} . De kracht van deze techniek is zijn eenvoudigheid. Er hoeven geen ingewikkelde signalen verwerkt te worden, enkel het aan of uit signaal van de actuator. Een groot nadeel van deze techniek is dat de sensor fysiek tegen een object aan moet komen om het te detecteren. In een zwerm van relatief fragiele robots is dit niet ideaal. Wanneer de robots zich op ruiger terrein bevinden, zal het gebruik van een "bumper" of arm ook niet ideaal zijn, deze kan namelijk vast blijven zitten tegen of achter obstakels.



\subsubsection{Sensor choice for the robot}
Now the question is which of the different sensor techniques proposed, is the best to implement on the swarm robots. While answering this question the following points will be taken into account:

\begin{itemize}
    \item The sensor must be easy to reproduce because multiple robots must be made from the swarm.
    \item The budget is limited, therefore the sensor price should be cheap compared the the total price of one robot.
    \item Energy usage should be as low as possible to increase the operation time per robot.
    \item For budget reasons there has been chosen to not take into account if the sensor would work on Mars.
\end{itemize}

Looking at these points it becomes clear that the lidar sensor wouldn't be the best choice. It doesn't really have any outstanding pros compared to the other two techniques and it's much more expensive. Currently the cheapest lidar sensor would cost about 100$\euro$. The ultrasonic and mechanical sensor wouldn't cost more than a few Euro's. As talked about before the mechanical sensor has a few drawbacks compared to the ultrasonic sensor. For these reasons the ultrasonic sensor seems the best choice. The only issue left to solve is how to sense the difference between obstacles which the robot has to avoid, and things the robot doesn't have to avoid, but still might be detected by the sensor. This will be discussed in the next paragraph.

\subsubsection{Obstacle discrimination with an ultrasonic sensor}
An ultrasonic sensor is usually used to detect if there is some obstacle in the way. In this case there is made no difference made between a obstacle that the robot can move over, or a obstacle that should be avoided. For example: a small slope or hill might be detected as an obstacle, but in truth the robot can move over it. The robots developed for this program will eventually be able to cross harsh terrain. In this case the proximity sensor will need to discriminate between these situations. Working with ultrasonic waves there are multiple domains to work with. These are time, frequency and amplitude. Time is used to determine the distance between the object and the sensor. The change in frequency and/or amplitude contains information about the shape of the object\ref{ultraobject}. 
In the paper "Object recognition with ultrasonic sensor" is concluded 
that looking at the amplitude over time contains enough information to discriminate between object\ref{ultraobject}. The example is given of an object with separated surfaces of 3,5cm, the echo from the second surface will be delayed by 0,2msec relative to the echo from the first surface. This delay can be easily detected with a microprocessor. The reason looking at frequency isn't the first choice is because of the higher hardware requirements it would require to properly sample the signal. 
\\

\newpage

\section{Bibliografie}
\bibliography{references}
\bibliographystyle{IEEEtran}



\end{document}