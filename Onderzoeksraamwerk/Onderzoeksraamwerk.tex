\section{Problem analysis}
Researcher have always been inspired by nature. When they looked at "social" insects like ants they discovered "swarming"\cite{swarmwiki} . The behaviour of one ant on its own seems illogical, but together they solve problems of great importance for the entire colony. These ants make us of the so called "trail laying" and "trail following" principle. Every ants lays a trail of pheromones, when a few ants walk back and forth to a food source. The one walking the shortest route will lay a more concentrated trail. The other ants will get attracted by the strongest trail, this way a positive feedback loop is created, which will make every ant walk the shortest route if given enough time. This is one example where relatively simple units, can achieve complex goals like path finding because they work together in a swarm. This principle is called "swarm intelligence"\cite{swarmintelligence}.

Swarming can have alot of up sides compared to the "classical" approach. A few are: quicker solution time, lower unit complexity and a greater fault tolerance\cite{swarmintelligence}. When for example one of the units breaks down, then will the other units still be able to complete the task. When this happens to one, more complex unit, this wont be the case. For these reasons its interesting to researching the applications of swarming.



\section{Context analysis}
The Delft university of technology started a program to research the applications of swarming. In this program multiple universities work together to make this research possible. The idea of this program is that each project group contributes a small bit to reach the end result, which will be a working swarm of robots. The technology developed should be modular, so it can be easily used on other platforms.\\The programs focus lies on researching the applications of swarming on Mars. Its preferable that the technologies developed also work on Mars, but in some cases other technologies can be used to cut the cost. For example, for a simple proximity sensor on earth, an ultrasonic sensor would do just fine. But in the Mars atmosphere the ultrasonic waves get heavily dampened to the point where the sensor just wont work\cite{soundonmars}. The cheapest sensor that would work on Mars would be a lidar\cite{lidarmars}. While the average ultrasonic sensor costs around 2$\euro$ the cheapest lidar costs atleast 100$\euro$. In this project the aim is to build one unit for around 200$\euro$, just one lidar sensor would be half of the robots budget. To proof the concept of swarming the robots don't need to be "Mars proof", so costs will be cut where possible.\\

\section{Problem definition}
Swarming intelligent systems are typically made up of simple agents(robots) interacting locally with one another and their environment. The group of individuals acting in such manner is referred to as a swarm\cite{swarmintelligence}. For a group of robots to qualify as a swarm-robotics the following criteria have to be met:

\begin{itemize}
	\item Autonomy � It is required that the individuals that make up 	the swarm-robotic system are autonomous robots. They are able to 		physically 		interact with the environment and affect it.
	\item Large number � A large number of units is required
	as well, so the cooperative behavior (and
	swarm intelligence) may occur. The minimum number
	is hard to define and justify. The swarm-robotic
	system can be made of few homogeneous groups of
	robots consisted of large number of units. Highly heterogeneous
	robot groups tend to fall outside swarm
	robotics.
	\item Limited capabilities � The robots in a swarm
	should be relatively incapable or inefficien on their
	own with respect to the task at hand.
	Scalability and robustness � A swarm-robotic
	system needs to be scalable and robust. Adding the
	new units will improve the performance of the overall
	system and on the other hand, loosing some units will
	not cause the catastrophic failure.
	\item Distributed coordination � The robots in a swarm
	should only have local and limited sensing and communication
	abilities. The coordination between the
	robots is distributed. The use of a global channel for
	the coordination would influence the autonomy of the
	units.
\end{itemize}

These criteria are a good indication of what makes a system swarm-robotic. But should not be used to determine whether a system is swarm-robotic or not. This is because some criteria are still somewhat phage\cite{swarmintelligence}.

Looking at these criteria we chose to define the problem into two sub-problems. One covers Autonomy, Large number and limited capabilities. And the other Distributed coordination.

There are multiple aspects that should be taken into account. First the project focus will be building a swarming module. The features of this module will be discussed later. What's important is that the module should be "plug and play". What is meant by this is that the only in and output will be a communication protocol and a power input. This way the module can later be used on different platforms later in the program.\\ Then To properly test the swarming module, a swarm of robots will be need to be developed.

\subsection{Doelstelling}
In samenwerking met de technische universiteit Delft en de Hogeschool van Amsterdam moet er een zwerm robots (Swarm) worden ontwikkeld die samen een taak uitvoeren. Swarming is een zeer breed begrip. Daarom wordt hieronder opgesomd, aan welke specificaties voldaan moet worden voordat er over swarming gesproken mag worden.Tevens zal hieruit een duidelijk doel naar voren komen wat behaald moet worden wat betreft het laten samen werken van verschillende robots binnen een zwerm.

Een groep robots moet aan de volgende eisen voldoen om onder de catogorie swarming te vallen:
