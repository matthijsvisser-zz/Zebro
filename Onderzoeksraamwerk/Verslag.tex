\documentclass[10pt,a4paper]{article}
\usepackage[utf8]{inputenc}
\usepackage{amsmath}
\usepackage{amsfonts}
\usepackage{amssymb}
\usepackage{graphicx}
\usepackage{color}
\usepackage{float}
\usepackage[dutch]{babel}
\usepackage{eurosym}
 
\definecolor{codegreen}{rgb}{0,0.6,0}
\definecolor{codegray}{rgb}{0.5,0.5,0.5}
\definecolor{codepurple}{rgb}{0.58,0,0.82}
\definecolor{backcolour}{rgb}{0.95,0.95,0.92}
\usepackage{listings}
\lstdefinestyle{cstyle}{
    backgroundcolor=\color{backcolour},   
    commentstyle=\color{codegreen},
    keywordstyle=\color{magenta},
    numberstyle=\tiny\color{codegray},
    stringstyle=\color{codepurple},
    basicstyle=\footnotesize,
    breakatwhitespace=false,         
    breaklines=true,                 
    captionpos=b,                    
    keepspaces=true,                 
    numbers=left,                    
    numbersep=5pt,                  
    showspaces=false,                
    showstringspaces=false,
    showtabs=false,                  
    tabsize=2
}
\graphicspath{ {./images/} }

\begin{document}
\begin{titlepage}
    \centering
    \vfill
    {\Large
    Swarming\\
   
    {\small Research framework}\\
        
        \vskip2cm
        {\small M. van Wilgenburg, W. Mukhtar, E. van Splunter, M. Siekerman, T. Zaal and M. Visser}\\
    }    
    \vfill
%    \includegraphics[width=1\textwidth]{WireS4}
    
    \vfill
    \vfill
\end{titlepage}

\newpage

\tableofcontents
\newpage


\section{Problem analysis}
Researcher have always been inspired by nature. When they looked at "social" insects like ants they discovered "swarming"\cite{swarmwiki} . The behaviour of one ant on its own seems illogical, but together they solve problems of great importance for the entire colony. These ants make us of the so called "trail laying" and "trail following" principle. Every ants lays a trail of pheromones, when a few ants walk back and forth to a food source. The one walking the shortest route will lay a more concentrated trail. The other ants will get attracted by the strongest trail, this way a positive feedback loop is created, which will make every ant walk the shortest route if given enough time. This is one example where relatively simple units, can achieve complex goals like path finding because they work together in a swarm. This principle is called "swarm intelligence"\cite{swarmintelligence}.

Swarming can have alot of up sides compared to the "classical" approach. A few are: quicker solution time, lower unit complexity and a greater fault tolerance\cite{swarmintelligence}. When for example one of the units breaks down, then will the other units still be able to complete the task. When this happens to one, more complex unit, this wont be the case. For these reasons its interesting to researching the applications of swarming.



\section{Context analysis}
The Delft university of technology started a program to research the applications of swarming. In this program multiple universities work together to make this research possible. The idea of this program is that each project group contributes a small bit to reach the end result, which will be a working swarm of robots. The technology developed should be modular, so it can be easily used on other platforms.\\The programs focus lies on researching the applications of swarming on Mars. Its preferable that the technologies developed also work on Mars, but in some cases other technologies can be used to cut the cost. For example, for a simple proximity sensor on earth, an ultrasonic sensor would do just fine. But in the Mars atmosphere the ultrasonic waves get heavily dampened to the point where the sensor just wont work\cite{soundonmars}. The cheapest sensor that would work on Mars would be a lidar\cite{lidarmars}. While the average ultrasonic sensor costs around 2$\euro$ the cheapest lidar costs atleast 100$\euro$. In this project the aim is to build one unit for around 200$\euro$, just one lidar sensor would be half of the robots budget. To proof the concept of swarming the robots don't need to be "Mars proof", so costs will be cut where possible.\\

\section{Problem definition}
Swarming intelligent systems are typically made up of simple agents(robots) interacting locally with one another and their environment. The group of individuals acting in such manner is referred to as a swarm\cite{swarmintelligence}. For a group of robots to qualify as a swarm-robotics the following criteria have to be met:

\begin{itemize}
	\item Autonomy - It is required that the individuals that make up 	the swarm-robotic system are autonomous robots. They are able to 		physically 		interact with the environment and affect it.
	\item Large number - A large number of units is required
	as well, so the cooperative behavior (and
	swarm intelligence) may occur. The minimum number
	is hard to define and justify. The swarm-robotic
	system can be made of few homogeneous groups of
	robots consisted of large number of units. Highly heterogeneous
	robot groups tend to fall outside swarm
	robotics.
	\item Limited capabilities - The robots in a swarm
	should be relatively incapable or inefficien on their
	own with respect to the task at hand.
	\item Scalability and robustness - A swarm-robotic
	system needs to be scalable and robust. Adding the
	new units will improve the performance of the overall
	system and on the other hand, loosing some units will
	not cause the catastrophic failure.
	\item Distributed coordination - The robots in a swarm
	should only have local and limited sensing and communication
	abilities. The coordination between the
	robots is distributed. The use of a global channel for
	the coordination would influence the autonomy of the
	units.
\end{itemize}

These criteria are a good indication of what makes a system swarm-robotic. But should not be used to determine whether a system is swarm-robotic or not. This is because some criteria are still somewhat vague\cite{swarmintelligence}.Looking at these criteria we chose to define the problem into two sub-problems. This division is chosen because problem one will cover distributed coordination and scalability with what we call a "swarming module", which will provide communication and relative localization to the other units in the swarm. And problem two covers Autonomy, Large number and limited capabilities.  Problem two will cover the platform that will provide movement and some basic autonomous functions. The projects focus will be the swarming module. The swarming module has to be completely modular so it can easily be used on other robotic platforms in future projects. Because of this the commissioning party stresses that this has to be a finished product. Resources(time) should be focused on the swarming module. For the sake of demonstration, problem two (the robot platform) will also need to be developed. This platform will have to provide some basic autonomous functions like movement and obstacle avoidance. One important requirement is that it should be easily replicated, so a swarm of these platforms can be created.\\

\subsection{Goal}
As previously discussed the problem is now divided in two sub-problems with the focus on the swarming module. In this paragraph will be discussed what will be expected from the swarming module and the robot platform.

\subsubsection{Swarmingmodule}
Distributed coordination is one of the swarming criteria. To achieve this, some form of (relative) localization is needed. This should keep units from moving to close or to far from each other. And could also be used to accomplish certain goals like;  mapping, assembly of structures or inspections \cite{networkedRS}. Communication is needed to share information about the environment and every units position. Requirements here are that the communication should not be depended on one host and should be scalable. This is so communication is not cut off when one of the units breaks down. The scale-ability is important so that units can be added or removed from the swarm \cite{multidomaincom}. Because this project is part of a running program the modules made should be modular so they can be used on future projects. Summed up, the swarming module has the following characteristics.

\begin{itemize}
\item (Relative) Localization
\item Communication
\item Scalable
\item Modular
\end{itemize}

\subsubsection{Robot platform}
To properly demonstrate and test the swarming module, a robotic platform will be needed. This platform will need to meet the swarming criteria of autonomy, large numbers and limited capabilities. The autonomy of the robot should stay simple and robust, the key features are movement, obstacle avoidance and power supply. The robot has to be easily duplicated so multiple units can be produced for the swarm. Also it should be able to house the swarming module.

\section{Research-question}

The main question of this research is as following: \textit{"How can communication and relative localization be achieved in a swarm or robots?".} To give an answer to this question there are multiple sub-questions to research first. 
 


\section{Sub-questions} 
The following questions need to be answered to come to a good conclusion to our research:

\begin{itemize}

    \item "What is swarming?"
    \begin{itemize}
        \item "What is the definition of swarming?"
        \item "How many robots are needed to create a swarm?"
        \item "How do the units communicate within the swarm?"
        \item "How do robots in the swarm know their location?"
    \end{itemize}    
    \item "Swarming communication"
    \begin{itemize}
        \item "What software protocol should be used?"
        \item "What hardware protocol should be used?"
        \item "What is de minimal required communication speed?"
        \item "What hardware is needed to implement the communication?"
    \end{itemize}
        \item "Communication between modules"
    \begin{itemize}
        \item "What software protocol should be used?"
        \item "What hardware protocol should be used?"
        \item "What is de minimal required communication speed?"
        \item "What hardware is needed to implement the communication?"
    \end{itemize}
    \item "What methods are there to propel the robot?"
    \begin{itemize}
        \item "What actuators can be used to drive the robot?"
        \item "How would the energy of the actuator be used to make the robot drive?"
        \item "What kind of effectors should be used?"
    \end{itemize}
    \item "What energy supply should be used to distribute the energy in the robot?"
    \begin{itemize}
        \item "How can the energy supply be managed?"
        \item "Is it possible to implement a recharge point?"
        \item "How can the energy supply be handled safely?"
    \end{itemize}
    \item "How will autonomous obstacle avoidance be achieved?"
    \begin{itemize}
   		 \item "What types of sensor technigues are there?"
    \end{itemize}
\end{itemize}
\newpage

\section{Specificaties}
In dit document worden de functionele eisen omschreven voor het bouwen van een robot die in samenwerking met andere robots een taak kan uitvoeren, zwermen. De specificaties zijn opgedeeld in verschillende functies waarmee het doel bereikt kan worden. 
Volgens de MoSCow -methode worden de te behalen specificaties op het gebied van zwermen geclassificeerd. De bijbehorende specificaties worden in dit hoofdstuk verder toegelicht. In het blokschema getoond in figuur \ref{fig:blockschematic}, zijn de verschillende modules aan elkaar verbonden. 

\begin{figure}[h]
    \centering
    \includegraphics[width=1\textwidth]{blockschematic}
    \caption{Versimpelde weegave in een blokschema van de samenhang van de verschillende deelsystemen binnen de robot.}
    \label{fig:blockschematic}
\end{figure}



\subsection{Must haves}
\begin{itemize}
\item Robots moeten onderling kunnen communiceren
\item De communicatie moet draadloos zijn
\item Iedere module moet zijn eigen energievoorziening hebben
\item De robot moet zichzelf omridirectioneel voort kunnen bewegen
\item Kan met meerdere robots opereren in een zwerm
\end{itemize}

\subsection{Should haves}
\begin{itemize}
\item In staat om een payload te vervoeren
\item Modulaire opbouw van systeem
\end{itemize}

\subsection{Could haves}
\begin{itemize}
\item Kan zich voortbewegen volgens het principe van de ZebRo
\item Mogelijkheid tot zelfstandig opladen/bijladen van energievoorziening
\item Toont een vorm van intelligentie
\item Schaalbaar maken van de zwerm populatie
\item Is er zich van bewust wanneer de robot zich ondersteboven bevind
\item De robot kan zich voortbewegen volgens het principe van de robot
\item De robots moeten modulair zijn
\item De robots kunnen zich opladen in een oplaadstation
\end{itemize}

\subsection{Won't haves}
\begin{itemize}
\item Een master in het zwerm netwerk
\end{itemize}

\newpage

\section{Deelspecificaties}

\begin{table}[H]
\centering
\caption{Sensoren}
\label{Sensoren}
\begin{tabular}{|l|l|}
\hline
Module    & Sensoren \\ \hline
Ingangen  & Communicatie met het systeem             \\ \hline
Uitgangen & Communicatie met het systeem             \\ \hline
Functies   & \begin{tabular}[c]{@{}l@{}}Meet de afstand tussen zichzelf en de omgeving\\ Moet de gemeten waarde communiceren naar de controller\end{tabular}            \\ \hline
\end{tabular}
\end{table}

De sensoren aanwezig op de robot moeten awareness creëren van zijn omgeving zodat hij zich hierop kan anticiperen. Zie tabel \ref{Sensoren}\\

\begin{table}[H]
\centering
\caption{Voeding (status)}
\label{voeding}
\begin{tabular}{|l|l|}
\hline
Module    & Voeding (status) \\ \hline
Ingangen  & Voeding            \\ \hline
Uitgangen & Communicatie met het systeem             \\ \hline
Functies   & \begin{tabular}[c]{@{}l@{}}Meet de spanning/stroom van de voeding\\ Moet de gemeten waarde communiceren naar de controller\end{tabular}            \\ \hline
\end{tabular}
\end{table}

Het is belangrijk om het accuniveau te meten om te voorkomen dat een robot uit de zwerm stil komt te staan. De gemeten waarde wordt gecommuniceerd naar de controller. Zie tabel \ref{voeding}\\

\begin{table}[H]
\centering
\caption{Draadloze communicatie module}
\label{communicatie}
\begin{tabular}{|l|l|}
\hline
Module    & Draadloze communicatie module \\ \hline
Ingangen  & \begin{tabular}[c]{@{}l@{}}Communicatie met het systeem\\ Communicatie met de andere robots in de zwerm          
\end{tabular}   \\ \hline
Uitgangen & \begin{tabular}[c]{@{}l@{}}Communicatie met het systeem\\ Communicatie met de andere robots in de zwerm          
\end{tabular}   \\ \hline
Functies   & \begin{tabular}[c]{@{}l@{}}brengt de communicatie tussen de zwerm modules tot stand\\ Verkrijgt informatie betreft de afstand tussen de verschillende robots\end{tabular}            \\ \hline
\end{tabular}
\end{table}

Om te communiceren tussen verschillende units moet er een communicatienetwerk worden opgesteld met \'e\'en gemeenschappelijk protocol. Daarnaast moet het protocol ook extra gegevens kunnen verzorgen zoals locatievoorziening. Zie tabel \ref{communicatie} \\

\begin{table}[H]
\centering
\caption{Effectoren}
\label{Effectoren}
\begin{tabular}{|l|l|}
\hline
Module    & Effectoren \\ \hline
Ingangen  & aansturing  \\ \hline
Uitgangen & Mechanische kracht   \\ \hline
Functies  & Drijft de robot aan en is in staat om bochten te maken op commando            \\ \hline
\end{tabular}
\end{table}

De aandrijving moet, naast dat het werk op in de testomgeving, ook goed kunnen functioneren op mars. Zie tabel \ref{Effectoren} \\

\begin{table}[H]
\centering
\caption{Centraal besturingsysteem}
\label{Centraal besturingsysteem}
\begin{tabular}{|l|l|}
\hline
Module    & Centraal besturingsysteem \\ \hline
Ingangen  & Alle informatie vanuit het systeem  \\ \hline
Uitgangen & Alle aansturing van de modules in de robot \\ \hline
Functies  & Hier komt alle communicatie samen en regelt de in- en uitgangen            \\ \hline
\end{tabular}
\end{table}

Het centraal besturingsysteem verzorgt alle communicatie tussen de verschillende modules die aanwezig zijn op de robot. Zie tabel \ref{Centraal besturingsysteem}

\newpage

\section{Bibliografie}
\bibliography{references}
\bibliographystyle{IEEEtran}



\end{document}